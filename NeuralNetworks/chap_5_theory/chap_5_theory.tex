\documentclass{article}
\usepackage{amsmath}

\title{Chapter theory 5 solution}
\date{2016-09-27}
\author{Artem Puzanov}


\begin{document}

\maketitle
\pagenumbering{gobble}
\newpage
\pagenumbering{arabic}


\section{Unstable gradient for simple networks}
\subsection{Futility of different activation value}
\paragraph{Intuition}
The core problem of vanishing/exploding gradient is not value per se, but chain rule. 
Any consistent increasing/decreasing effect is bound to become a problem if the chain is long enough.
Some combinations of weight initialization and activation functions might be able to postpone the problem.
The longer the chain, the harder it will be to keep this balance though.

No mathematical notation needed

\subsection{Numerical bounds of weighted derivative}
\paragraph{Intuition}
Nothing hard here, just old tricks
\paragraph{Mathematical notation}

We want to prove that:
$$|w\sigma'(wa+b)| > 1$$
if $|w|> 4$. To do that, let's remember that $\sigma'(z) = \sigma(z)(1 - \sigma(z))$.
Maximum value for $x(1-x)$ is $0.25$. So 
$$|w 0.25| > 1$$
For that to be true, $|w| > 4$.

\subsection{Activation bounds}
\paragraph{Intuition}
Again, nothing interesting
\paragraph{Mathematical notation}
Using derivative for $\sigma$:
$$\sigma'(z) = \frac{e^(-wa-b)}{1 + e^{-wa-b}}$$
Setting $e^{-wa-b} = x$, we get:
$$w \frac{x}{1-x} >= 1$$
Solving that for $x$ and assuming $w > 4$ we get:
$$\frac{(|w|-2)-\sqrt{w^2-4|w|}}{2} \leq x \leq \frac{(|w|-2)+\sqrt{w^2-4|w|}}{2}$$
Using $e^{-wa-b} = x$, and making some changes to suit the answer better:
$$\frac{|w|\left(1-\sqrt{1-4/|w|}\right)}{2}-1 \leq e^{-wa-b} \leq \frac{|w|\left(1+\sqrt{1-4/|w|}\right)}{2}-1$$
Solving for $a$:
$$\frac{1}{|w|}\ln\left(\frac{|w|\left(1-\sqrt{1-4/|w|}\right)}{2}-1\right) - \frac{b}{|w|}\leq a \leq \frac{1}{|w|}\ln\left(\frac{|w|\left(1+\sqrt{1-4/|w|}\right)}{2}-1\right)-\frac{b}{|w|}$$

Taking a difference between the larger nad the smaller value (and simplyfing again) gives us the desired interval:
$$\frac{2}{|w|} \ln{\left(\frac{|w|(1+\sqrt{1-4/|w|})}{2}-1\right)}$$

\subsection{Sigmoid neuron to identity neuron}
\paragraph{Intuition}
The key idea is to creatively use coefficient, so that on $0 <= x <= 1$ our function behaves linearly.
To do that, we need to make $f'(x)dx = dx$ for our neuron, while keeping $f''(x), f'''(x)$ and so on close to zero.
\paragraph{Mathematical notation}
The first trick is to center our neuron on ${0.5, 0.5}$ coordinates.
That can be done with setting $w_1 = -2b$, so that $1/(1+e^{w_1x + b}) = 1/2$.
(Note: switchting signs in exponent is to save time on all that sign acrobatic)
Let's write down taylor expansion to the second order:
$$f(a) \approx f(a) + f'(a)(x-a) + \frac{f''(a)}{2} (x-a)^2$$
For our case that shall be
$$\frac{w_2}{1 + e^{w_1a+b}} - \frac{w_1w_2 e^{w_1a+b}}{(e^{w_1a+b} + 1)^2} (x-a) + \frac{w_{1}^{2} w_2 e^{w_1a+b} (e^{w_1a+b} - 1) }{(e^{w_1a+b} + 1)^3} (x-a)^2$$
This unholy thing is actually pretty mallable, once we remember that $w_1$ and $b$ are small.
We need second derivative to go to zero - for that $w_2 < 1/w_{1}^2$ is enough (check the limits if you want, though it's obvious)
Remembering that $w_1$ and $b$ are small, we get:
$$\frac{w_2}{2} - \frac{w_1 * w_2 (x - a)}{4}$$
Now, let's try to make sure that his function is equal to $x$ around $a = 1/2$:
$$\frac{w_2}{2} - \frac{w_1*w_2 (x - 1/2)}{4} = \frac{w_2}{2} + \frac{w_1 w_2}{8} - w_1w_2x/4$$
By setting $w_2 = -4/w_1$, we can get:
$$-2/w_1 -1/2 + x = x - 2/w_1 - 1/2$$.
Now, a note: this is definetly not just $x$, as it's tilted by a constant. So it seems I haven't solved the original :(
But! if we assume we can not only use $w_2$, but also have $b_2$, and set $b_2 = 2/w_1 + 1/2$,
than we get a perfect $x$.
NOTE: return here later.

\end{document}