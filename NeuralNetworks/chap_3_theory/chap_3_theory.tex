\documentclass{article}
\usepackage{amsmath}

\title{Chapter theory 3 solution}
\date{2016-09-27}
\author{Artem Puzanov}


\begin{document}

\maketitle
\pagenumbering{gobble}
\newpage
\pagenumbering{arabic}

\section{activation derivative proof}
\paragraph{Intuition}
Taking derivatives from exponents rather often ends in a kind of semi-recursive definitions, which come in handy for computational purposes.
\paragraph{Mathematical notation}
We want to prove that:
$$\sigma'(z) = \sigma(z)(1 - \sigma(z))$$
where $\sigma(z) = \frac{1}{1+e^{-z}}$
This is a rather simple derivation, using chain rule:
$$\sigma'(z) = \frac{e^{-z}}{(1 + e^{-z})^2}$$
Let's take out $\sigma(z)$
$$\sigma'(z) = \frac{1}{(1 + e^{-z})}\frac{e^{-z}}{1+e^{-z}}$$
$$\sigma'(z) = \sigma(z)\frac{e^{-z} + 1 - 1}{1+e^{-z}}$$
$$\sigma'(z) = \sigma(z)(1 - \frac{1}{1+e^{-z}})$$
$$\sigma'(z) = \sigma(z)(1 - \sigma(z))$$
Q.E.D.
\end{document}