\documentclass{article}
\usepackage{amsmath}

\title{Chapter theory 3 solution}
\date{2016-09-27}
\author{Artem Puzanov}


\begin{document}

\maketitle
\pagenumbering{gobble}
\newpage
\pagenumbering{arabic}


\section{Proof of universal computability}
\subsection{Solve the problem for 2 variables}

\subsubsection{Step wise function in any direction}

TODO: use orthogonal $\epsilon$, not just $y$

\paragraph{Intuition}
To prove that we can generate step-function in any direction, we will show that we can
generate steps orthogonal to ${x_1, x_2, ..., x_n}$ plane such that on the one side of the plane value of activation is close to $1$
and on the other it's close to $0$. For $2$ variables ($x$, $y$), this translates into a ${x, y}$ line, on the one side of which $\sigma(z)$
is $1$, and on the other it's $0$.
\paragraph{Mathematical notation}
Let's define ${x_1, x_2, ..., x_n}$ as $X$. We shall take arbitrary vector of coefficients $K = {k_1, k_2, ..., k_n}$.
Than our arbitrary plane can be written as $K \odot X = C$. This means that we have $n-1$ free variables, and one that is a linear combination of them.
$$x_j = C/k_j + \frac{<K_{-j} , X_{-j}>}{k_j}$$
where $K_{-j} , X_{-j}$ refer to vectors $K$ and $X$ withouth $j$-th element.

Now, what's a step outside this plane?
Well, it's the parralel plane (don't call me Sirely):
$$K \odot X = C + \epsilon$$
where $\epsilon$ is the size of our step.
Note that this translates into
$$x_j = \frac{C + \epsilon}{k_j} + \frac{<K_{-j} , X_{-j}>}{k_j}$$
This allows us to put all those coefficents into activation function.
For the statement to be true, we have to prove that these two inequalities hold:
$$\forall \delta \in (0, 0.5), \epsilon > 0 , C \in R,  k \in R: \exists W:$$
$$(1) \frac{1}{1 + exp(- \sum_{i \neq j}w_ix_i + w_j (\frac{C + \epsilon}{k_j} + \frac{<K_{-j} , X_{-j}>}{k_j})} > 1-\delta$$
$$(2) \frac{1}{1 + exp(- \sum_{i \neq j}w_ix_i + w_j (\frac{C - \epsilon}{k_j} + \frac{<K_{-j} , X_{-j}>}{k_j})} < \delta$$



\end{document}