\documentclass{article}
\usepackage{amsmath}

\title{Chapter theory 3 solution}
\date{2016-09-27}
\author{Artem Puzanov}


\begin{document}

\maketitle
\pagenumbering{gobble}
\newpage
\pagenumbering{arabic}


\section{Proof of universal computability}
\subsection{Solve the problem for 2 variables}

\subsubsection{Step wise function in any direction}

TODO: use orthogonal $\epsilon$, not just $y$

\paragraph{Intuition}
To prove that we can generate step-function in any direction, we will show that we can
generate lines on ${x, y}$ plane such that on the one side of the line value of activation is close to $1$
and on the other it's close to $0$. 
\paragraph{Mathematical notation}
Let's define line on ${x, y}$ plane as a line:
$$y = C + kx$$
where $C$ and $k$ are real numbers.
What we want to prove if this:
$$\forall \delta \in (0, 0.5), \epsilon > 0 , C \in R,  k \in R: \exists w_1, w_2$$
$$(1) \frac{1}{1 + exp(-(w_1x + w_2(C + kx + \epsilon) + b))} > 1 - \delta$$
$$(2) \frac{1}{1 + exp(-(w_1x + w_2(C + kx - \epsilon) + b))} < \delta$$

These statements essentially say this:
We can always find weights $w_1$ and $w_2$ as to generate a step along line $y = C + kx$.
$delta$ and $epsilon$ are there to describe what a step is.

Note that $<1 and >0$ conditions are satisfied due to the nature of activation function

Let's simplify $(1)$ first:

$$exp(-(w_1x + w_2(C + kx + \epsilon) + b)) < \frac{1}{1 - \delta} - 1$$
$$(w_1x + w_2(C + kx + \epsilon) + b) > -ln(\frac{1}{1 - \delta} - 1)$$
$$(w_1x + w_2(C + kx + \epsilon) + b) > ln(\frac{1 - \delta}{\delta})$$

Now let's simplify $(2)$:

$$exp(-(w_1x + w_2(C + kx - \epsilon) + b)) > \frac{1}{\delta} - 1$$
$$(w_1x + w_2(C + kx - \epsilon) + b) < -ln(\frac{1}{\delta} - 1)$$
$$(w_1x + w_2(C + kx - \epsilon) + b) < ln(\frac{\delta}{1 - \delta})$$

It's easy to see that right sides of $(1)$ and $(2)$ are negatives of each other.
Let's set $L = ln(\frac{1 - \delta}{\delta})$.  Note that $L$ is strictly positive, due to our condition on  $\delta$.
Putting that into our inequalities, we get:
$$(1) (w_1x + w_2(C + kx + \epsilon) + b) > L$$
$$(2) (w_1x + w_2(C + kx - \epsilon) + b) < -L$$

Combining those, we get:
$$w_2 > \frac{2L + b}{2 \epsilon}$$

As you can see, the result does not depend on $C$ and $k$, and we can always find $w_2$ that would satisfy this condition.
So we have proven that there exists such a $w_2$ that both conditions $(1)$ and $(2)$ will be satisfied.
Q.E.D.



\end{document}