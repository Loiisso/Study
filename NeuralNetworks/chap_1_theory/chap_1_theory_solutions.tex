\documentclass{article}
\usepackage{amsmath}

\title{Chaper theory 1 solution}
\date{2016-06-04}
\author{Artem Puzanov}


\begin{document}

\maketitle
\pagenumbering{gobble}
\newpage
\pagenumbering{arabic}

\section{Sigmoid neurons simulating perceptrons}
\subsection{part I}
\paragraph{Intuition:} 
Each neuron has a binary activation function. If we multiply both weights and biases by a positive
constant the activation value shall not change. If $w_ix_i$ and $b$ have different signs, it's obvious.
If they are both negative, neuron would show $0$ anyway. If they are both positive, neuron would show
$1$ anyway.
\paragraph{Mathematical notation:} 
Let's define $(a_{jl}$ value of activation function of $i$-th neuron in $l_th$ layer of our network
and $C$ as the multiplier constant ($C > 0$ )
\begin{equation*}
  a_{j} = \sum_{i}w_ix_i + b_j
\end{equation*}

If $$\sum_{i}w_ix_i + b_j > 0$$ then $$\sum_{i}Cw_ix_i + Cb_j > 0$$ as we can divide the second equation by $C$
to obtain the first equation. The case when $\sum_{i}w_ix_i + b_j < 0$ and $\sum_{i}w_ix_i + b_j = 0$ are the same.
As the choice of neuron was non-specific, this holds for all neurons in the network. 
The statement is proved.

\subsection{part II}
\paragraph{Intuition:} 
The activation function of the sigmoid neuron limits either at 0 or at 1 when $z$
is multiplied by a constant. Limit of the particular neuron depends on whether $wx + b > 0$ or $wx + b < 0$
\paragraph{Mathematical notaion} 
\begin{equation*}
  z_{j} = \sum_{i}w_ix_i + b_j
\end{equation*}
\begin{equation*}
  Cz_{j} = C\sum_{i}w_ix_i + Cb_j
\end{equation*}
Let's take arbtirary neuron $a_{jl}$. For perceptron, if $z > 0$ (and therefore $a = 1$) than $Cz > 0$ and $a(Cz)=1$. 
If it's a sigmoid neuron, than if $C -> \infty$ than $1/(1 + e^{-Cz}) -> 1$
If $z < 0$  (and therefore $a = 0$ than $Cz < 0$. For sigmoid neuron it's $C -> \infty$ than $1/(1 + e^{-Cz}) -> 0$

Therefore if $C -> \infty$ than for all $z < 0$ or $z > 0$ sigmoid neurons emulate perceptrons.

If $wx + b = 0$ than perceptron activation value is $0$, and activation value for this sigmoid neuron is
$1/(1 + e ^ 0) = 1/2$, which is different from perceptron activation value. Therefore this sigmoid neuron shall not emulate
it's perceptron counterpart

\end{document}

