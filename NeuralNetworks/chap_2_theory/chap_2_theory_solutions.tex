\documentclass{article}
\usepackage{amsmath}

\title{Chapter theory 2 solution}
\date{2016-08-27}
\author{Artem Puzanov}


\begin{document}

\maketitle
\pagenumbering{gobble}
\newpage
\pagenumbering{arabic}

\section{Matrix representations of main equations}
\subsection{part I}
\paragraph{Intuition}
Hadamard product is just an easy way to use vectors instead of diagonal matrices, not much thinking needed here )
\paragraph{Mathematical notation}
To prove this assertion, we must show that 
$$\delta^L = {\sum}'(z^L)\nabla_aC = \nabla_aC\odot\sigma'(z^L)$$
Let's call $\delta^L$ for non-Hadamard case $\delta'^L$.
Or equivalently, we must prove that $$\forall i: \delta^L_i = \delta'^L_i$$,
where $$\delta^L_i = \sigma'(z_i^L)\frac{\partial C}{\partial a_i}$$
Let's write $\delta'^L_i$ in index hell notation. 
A small note: I shall use $I^{\delta}$ as replacement for $\sum$ notation, as I shall user $\sum$ in it's usual arithmetical sum meaning further
$$\delta'^L_j = {\sum}_{j}I^{\delta}_{ij}\frac{\partial C}{\partial a_j}$$
By definition, $I^{\delta}_{i\neqj} = 0$, therefore:
$${\sum}_{j}I^{\delta}_{ij}\frac{\partial C}{\partial a_j} = I^{\delta}_{ii} \frac{\partial C}{\partial a_i}$$
Replacing $I^{\delta}_{ii}$ with $\sigma'(z_i^L)$ we get 
$$\delta'^L_i = \sigma'(z_i^L)\frac{\partial C}{\partial a_i}$$
Which is what we wanted to prove;

\subsection{part II}
\paragraph{Intuition}
Again, nothing hard here. Just prove for any $j$ that result is equivalent to Hadamar project
\paragraph{Mathematical notation}


To prove this assertion, we must show that 
$$\delta^l = {\sum}'(z^l) (w^{l+1})^{T} \delta^{l+1} = (w^{l+1})^{T} \delta^{l+1}\odot\sigma'(z^l)$$

Let's call $\delta^l$ for non-Hadamard case $\delta'^l$.

Or equivalently, we must prove that $$\forall i: \delta^l_i = \delta'^l_i$$,
where $$\delta^l_i = \sigma'(z_i^l)\frac{\partial C}{\partial a_i}$$

A small note: I shall use $I^{\delta}$ as replacement for $\sum$ notation, as I shall user $\sum$ in it's usual arithmetical sum meaning further:

For non-hadamar case
$$\delta'^l_i = \sum_k I^{\delta}_{ki} \sum_j(w^{l+1})^T_{ij} \delta^{l+1}_j$$
This cluster... of indexes means that we take a row of I^{\delta}, and multiply it by dot product of (row  $w^T$) and ($\delta^{l+1}_j$)
By definition, $I^{\delta}_{i \neq j} = 0$, therefore:



(STOPPED HERE)





\end{document}
